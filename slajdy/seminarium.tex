\documentclass[a4paper]{beamer}

\usepackage{polski}
\usepackage[utf8]{inputenc}
\usepackage{amsmath}
\usepackage{amssymb}
\usepackage{enumerate}
\usepackage{hyperref}
\usepackage{listings}
\usepackage{graphicx}
\usepackage{color}

\graphicspath{ {./images/} }
\usetheme{Warsaw}
\useoutertheme{infolines}
\setbeamertemplate{footline}{}
\setbeamertemplate{headline}
{
\begin{beamercolorbox}{section in head/foot}
\vskip2pt\insertnavigation{\paperwidth}\vskip2pt
\end{beamercolorbox}%
}

\DeclareGraphicsExtensions{.png}

\author{Agnieszka Pocha, Michał Kowalik}
\title{Klątwa wielowymiarowości \\ The Curse of Dimensionality}
\date{11 marca 2015}

\begin {document}


\begin{frame}
\titlepage
{\footnotesize
na podstawie książki: \\
Bertrand Clarke, Ernest Fokoue, Hao Helen Zhang \\
}
\textit{Principles and Theory for Data Mining and Machine Learning}
\end{frame}


\begin{frame}
\frametitle{Agenda}
\tableofcontents
\end{frame}

\section{AI, ML, Data Mining}
\begin{frame}
\begin{block}{Sztuczna Inteligencja - AI}
\end{block}

\begin{block}{Uczenie Maszynowe - Machine Learning}

\end{block}

\begin{block}{Data Mining}

\end{block}
\end{frame}

\section{Local vs Global Methods}
\begin{frame}
\begin{block}{Local Methods}

\end{block}
\begin{block}{Global Methods}

\end{block}
\end{frame}

\section{The Curse}
\begin{frame}
\begin{block}{'Definicja'}

\end{block}
\end{frame}

\begin{frame}
\begin{block}{Sparsity}

\end{block}
\end{frame}
\begin{frame}
\begin{block}{Liczba modeli}

\end{block}
\end{frame}

\section{Metody rozwiązywania problemu}
\begin{frame}
\begin{block}{PCA}

\end{block}
\begin{block}{LDA}

\end{block}
\end{frame}
\end{document}