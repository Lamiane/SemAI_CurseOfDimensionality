\documentclass[a4paper]{beamer}

\usepackage{polski}
\usepackage[utf8]{inputenc}
\usepackage{amsmath}
\usepackage{amssymb}
\usepackage{enumerate}
\usepackage{hyperref}
\usepackage{listings}
\usepackage{graphicx}
\usepackage{color}

\graphicspath{ {./images/} }
\usetheme{Warsaw}
\useoutertheme{infolines}
\setbeamertemplate{footline}{}
\setbeamertemplate{headline}
{
\begin{beamercolorbox}{section in head/foot}
\vskip2pt\insertnavigation{\paperwidth}\vskip2pt
\end{beamercolorbox}%
}

\DeclareGraphicsExtensions{.png}

\author{Agnieszka Pocha \\ Michał Kowalik}
\title{Klątwa wielowymiarowości \\ The Curse of Dimensionality}
\date{11 marca 2015}

\begin {document}


\begin{frame}
\titlepage
{\footnotesize
na podstawie książki: \\
Bertrand Clarke, Ernest Fokoue, Hao Helen Zhang \\
}
\textit{Principles and Theory for Data Mining and Machine Learning}
\end{frame}


\begin{frame}
\frametitle{Agenda}
\tableofcontents
\end{frame}

\section{AI, ML, Data Mining}
\begin{frame}
\begin{block}{Sztuczna Inteligencja - AI}
\textit{W świecie, gdzie niepewność modelu jest często ograniczeniem w procedurach wnioskowania, ważniejszym stała się predykcja/przewidywanie niż testowanie czy estymacja.}... \\
Dział informatyki zajmujący się rozwiązywaniem problemów, które nie są efektywnie algorytmizowalne
\end{block}

\begin{block}{Uczenie Maszynowe - Machine Learning}
Pojęcie odnosi się do użycia formalnych struktur (maszyny) do wnioskowania (uczenie) - MODELOWANIE. Informacja tutaj pomaga zmniejszyć niepewność.
\end{block}

\begin{block}{Data Mining}
Odnosi się do przeszukiwania ogromnych, wielowymiarowych, wielotypowych zbiorów danych. Dane są nieustrukturyzowane i wielorakie.
\end{block}
\end{frame}

\begin{frame}
\begin{block}{Przestrzeń}
Przestrzeń – zbiór, w którym określone są rozmaite relacje i działania pomiędzy jego elementami
\end{block}
\begin{block}{Metryka}
Metryką (w zbiorze X) nazywa się funkcję: \\
$d: X \times X \to [0, +\infty),$
która dla dowolnych elementów a, b, c tego zbioru spełnia następujące warunki:
\begin{itemize}
\item identyczność nierozróżnialnych: $d(a, b) = 0 \iff a = b$
\item symetria: $d(a, b) = d(b, a)$
\item warunek trójkąta: $d(a, b) \leqslant d(a, c) + d(c, b)$
\end{itemize}
Gdy d jest metryką w zbiorze X, to para (X, d) nazywana jest przestrzenią metryczną
\end{block}
\end{frame}
\begin{frame}
\begin{block}{Metryka Euklidesowa}
Ogólnie, w przestrzeni $\mathbb R^n$ metrykę euklidesową definiuje się wzorem:
$$d_e(\mathbf x, \mathbf y) = \sqrt{(y_1 - x_1)^2 + \dots + (y_n - x_n)^2}$$
tzn. jako pierwiastek euklidesowego iloczynu skalarnego różnicy dwóch wektorów przez siebie:
$$d_e(\mathbf x, \mathbf y) = \sqrt{\langle \mathbf y - \mathbf x, \mathbf y - \mathbf x \rangle}$$
\end{block}
\begin{block}{Metryka Jaccarda}
Metryka używana do porównywania zbiorów:
$J(A,B) = {{|A \cap B|}\over{|A \cup B|}}$ \\
Żeby były spełnione warunki metryki:
$$d_J(A,B) = 1 - J(A,B) = { { |A \cup B| - |A \cap B| } \over |A \cup B| }$$
\end{block}
\end{frame}

\section{Local vs Global Methods}
\begin{frame}
\begin{block}{Local Methods}
- K-means
\end{block}
\begin{block}{Global Methods}
- sieci neuronowe
\end{block}
\end{frame}

\section{The Curse}
\begin{frame}
\begin{block}{'Definicja' - Curse of Dimensionality}
Rzeczywisty wymiar modelu jest skończony, ale rozmiar przestrzeni w której się znajduje może być nieograniczony. \\
Trudność szacowania rośnie w sposób wykładniczy względem wymiaru.
\end{block}
Ekstremalny przypadek problem jest, gdy duże $p$, małe $n$, gdzie $p$ - rozmiar przestrzeni, $n$ - ilość danych.
\begin{block}{Intuicja}
Przy wysokim wymiarze przestrzeni, dane są zbyt rzadkie. \\
Przy wysokim wymiarze przestrzeni, liczba możliwych modeli do rozważenia rośnie w sposób wykładniczy (superexponential).

\end{block}
\end{frame}

\begin{frame}
\begin{block}{Sparsity}

\end{block}
\begin{block}{Concurvity}

\end{block}
\begin{block}{Liczba modeli}

\end{block}

\end{frame}

\section{Sparsity}
\begin{frame}
\begin{block}{Motywacja}
Jeśli nie ma zbyt wiele obiektów do porównania w sąsiedztwie jakiegoś punktu $x$, wtedy trudno określić jak powinna wyglądać funkcja $f(x)$.
\end{block}
\begin{block}{}
Gdy liczba wymiarów $p$ rośnie, liczba danych lokalnych maleje do 0. 
\end{block}
\begin{block}{}
Objętość kuli o promieniu $r$ maleje do 0, wraz ze wzrostem wymiaru $p$. \\
$V_{n}=\frac { \pi^{\frac{n}{2}}}{\Gamma (\frac{n}{2}+1)}\cdot r^{n} = \begin{cases} \displaystyle {\pi^k\over k!}\cdot r^n & \mbox{dla }n=2k, \\[2ex] \displaystyle {2^k \pi^{k-1}\over n!!}\cdot r^n & \mbox{dla } n=2k-1, \end{cases}$
\end{block}
\end{frame}

\section{Concurvity}
\begin{frame}
\begin{block}{Concurvity}

\end{block}
\end{frame}


\section{Liczba modeli}
\begin{frame}
\begin{block}{Liczba modeli}

\end{block}
\end{frame}

\section{Metody rozwiązywania problemu}
\begin{frame}
\begin{block}{PCA}

\end{block}
\begin{block}{LDA}

\end{block}
\end{frame}
\end{document}